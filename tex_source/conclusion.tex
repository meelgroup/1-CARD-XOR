\section{Conclusion}\label{sec:conclusion}
In this paper, we study 1-CARD-XOR formulas, which are expressed as a conjunction of cardinality constraints and XOR constraints. The CARD-XOR formulas are ubiquitous in several domains of interest such as their close relationship to maximum likelihood decoding and their importance in hashing-based techniques for discrete integration. Our study revealed the empirical existence of phase transition region of satisfiability of random 1-CARD-XOR formulas for which we were able to establish tight theoretical bounds. 

The investigation into runtime behavior led to the surprising discovery of behavior reminiscent of random CNF formulas but significantly different from recent studies on CNF-XOR formulas. Furthermore, we observed that despite significant interest in CP/SAT communities devoted to design of encodings, the qualitative nature of runtime behavior remains consistent across different encodings. Finally, we discover a significant impact of branching heuristics on the runtime behavior. Similar to other CSP problems where the study of phase transition have led to development of algorithmic insights, our study opens future directions into the development of algorithmic techniques for efficient CARD-XOR solvers in practice. 
\section*{Acknowledgements}
This research is supported in part by the National Research Foundation Singapore under its AI Singapore Programme (Award Number: [AISG-RP-2018-005])., the NUS ODPRT Grant [R-252-000-
685-133] and SERB, DST, India through [ECR 2017/001126]. The computational work for this article was performed
on resources of the National Supercomputing Centre, Singapore. \url{https://www.nscc.sg/}.
