

\section{Introduction}

The study of runtime behavior of algorithmic techniques in the context of constraint satisfaction problems (CSP) has been key to several breakthroughs in the design of new solvers~\cite{DM94}. Specifically, a deep
connection was discovered between the density (ratio of the number of clauses to the number of variables) of random
propositional CNF fixed-width (fixed number of literals per clause) formulas and the runtime  behavior of SAT solvers on such formulas. 
For random $k$-CNF formulas, where every clause contains exactly $k$ literals, experiments
suggest a specific phase-transition density, for example 4.26 for random 3-SAT,
but establishing this analytically has been highly challenging~\cite{CP13a}, and it has been
established only for $k=2$~\cite{CR92} and all large enough $k$~\cite{DSS15}. 
A phase-transition phenomenon has also been identified in random XOR formulas (conjunctions of 
XOR constraints). 
Creignou and Daud{\'e}~\shortcite{CD99} proved a phase-transition at density 1 for variable-width 
random XOR formulas.

%KSM: SM19 is ApproxMC3 paper -- please add bib entry for it (see below)
 Recently, Dudek, Meel, and Vardi~\shortcite{DMV16,DMV17} extended such studies to the conjunction of CNF and XOR constraints, called CNF-XOR formulas. The motivation for their study was the usage of CNF-XOR formulas in the recent hashing-based techniques for the problem of propositional model counting~\cite{Stockmeyer83,CMV13b,CMV16,SM19}. 
 
 
 %Given a Boolean formula $F$, the problem of propositional model counting, also known as \#SAT, seeks to compute the number of solutions of $F$. 
 
  
  Satisfiability of  conjunction of a cardinality constraint and XOR constraints gives rise to an interesting problem, which we shall refer to as 1-CARD-XOR. Given a set of propositional variables, a cardinality constraint (CARD) puts  bounds on how many of these variables can be set to {\tt true}.
It is worth noting that 1-CARD-XOR is still NP-complete~\cite{BMT78}, even to approximate~\cite{ABSS93}, and as our experimental evaluation demonstrates, the study of 1-CARD-XOR alone is computationally expensive. Furthermore, 1-CARD-XOR formulas are necessary and sufficient for maximum likelihood decoding (MLD), one of the most crucial problems in coding theory, in which one seeks to extract the maximum amount of information from a noisy channel. The problem of maximum likelihood decoding is equivalent to determining satisfiability of a 1-CARD-XOR formula. Consequently, MLD has been subject to theoretical and practical investigations for over 50 years ~\cite{Chase85,TV15}.

%but the lack of usage of SAT/CSP solvers in the coding theory community has led to such a study not being undertaken. 

Generalization of a cardinality constraint is a Pseudo-Boolean (PB) constraint which enforces bounds on the summation of the weights of the propositional variables that can be set to {\tt true}.
A variant and a more generalized version of 1-CARD-XOR problem is the satisfiability of conjunction of CNF constraints, one Pseudo-Boolean constraint and random XOR constraints, denoted as CNF-PB-XOR 
formulas. Formulas of this kind play a crucial role in solving one of the fundamental problems in artificial intelligence: discrete integration.
  Given a set of constraints as a Boolean formula $F$ and a weight function $W$, the problem of discrete integration is to compute the total weight of the set of solutions of input constraints. This has applications in 
numerous areas, including probabilistic reasoning, machine learning, 
planning, statistical physics, inexact computing, and 
constrained-random verification
\cite{JS96,MP99,Bacchus2003,Sang04combiningcomponent,DH07,GSS08,Mur12,EGSS14a}.

Recently, two hashing-based approaches have been proposed for discrete integration: {\WISH}~\cite{EGSS13c} and {\WeightMC}~\cite{CFMSV14}.  Both of these approaches provide strong Probably Approximately Correct (PAC)-style guarantees, i.e., $(\varepsilon,\delta)$ guarantees. The core idea of {\WeightMC} is to partition the problem of discrete integration into linearly many {\em regions} such that the weight of satisfying assignments in each of the {\em regions} is {\em almost equal}; thereby allowing the usage of hashing-based unweighed counting techniques for each of the regions. Each of the regions can be represented by the conjunction of $F$ and one Pseudo-Boolean (PB) constraint. Consequently, the underlying SAT solver invoked during unweighted counting subroutine needs to handle the CNF-PB-XOR formulas.
 While the elegant formulation of {\WISH} and {\WeightMC} promises scalability and strong theoretical guarantees, {\WISH} and {\WeightMC} have not witnessed scalability similar to that of unweighted counting algorithms such as {\ApproxMCThree}. Unlike {\CryptoMiniSAT}, which is optimized for CNF-XOR formulas, to the best of our knowledge, there do not exist specialized solvers that can handle CNF-PB-XOR formulas efficiently. Design of solvers to efficiently  handle CARD-XOR formulas alone would push the boundaries of state-of-the-art techniques to handle several problems of interest~\cite{Par18}. 
 

The phase-transition behavior of CNF constraints has been analyzed to explain runtime behavior of {\SAT} solvers~\cite{AchCoj08}. Furthermore, the study of Dudek et al~\shortcite{DMV16,DMV17} contributed to the development of a new architecture for handling CNF-XOR constraints~\cite{SM19}. We believe that analysis of the phase-transition phenomenon for CARD-XOR formulas would be pivotal  towards demystifying the runtime behavior of the current state of the art solvers.  
Therefore, a deeper understanding of the runtime behavior of CSP/SAT solvers for CARD-XOR constraints can have far-reaching consequences. 


 The primary contribution of this work is the first rigorous empirical study to characterize the runtime behavior of 1-CARD-XOR formulas. In particular:
 \begin{enumerate}
 	\item We prove (in Section~\ref{sec:analysis}) upper and lower bounds on the location of the 1-CARD-XOR  phase-transition region.
 	
 	\item 	We present (in Section~\ref{sec:experiments}) experimental evidence for phase transition behavior of 1-CARD-XOR formulas, henceforth known as 1-CARD-XOR phase-transition that follows a non-linear trade-off between $k$-CNF clauses and XOR clauses.
 	
	\item  We demonstrate that the runtime behavior of SAT solver around phase transition is reminiscent of random CNF formulas but is surprisingly different from CNF-XOR formulas. This observation underscores the need for further exploration in this direction for deeper understanding. 

 	
 	
 \end{enumerate}
 
 The surprising nature of our observations opens up future directions of research and we hope that a deeper understanding would lead to the design of efficient CARD-XOR solvers in the future. The rest of the paper is organized as follows. We discuss notations and preliminaries in Section~\ref{sec:prelims}. We survey related work in Section~\ref{sec:related}. We present a theoretical analysis to obtain lower and upper bounds on the location of phase transition in Section~\ref{sec:analysis}. We then present the empirical behavior of 1-CARD-XOR constraints in Section~\ref{sec:experiments}.  We finally conclude in Section~\ref{sec:conclusion}. 


