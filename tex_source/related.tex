

\section{Related Work} \label{sec:related}
\subsubsection{Phase Transitions}
Motivated from statistical physics, the study of satisfiability of random constraint satisfaction problems led to the observation of a phase transition behavior~\cite{KS94}. In particular, the probability of satisfiability was observed to undergo a sharp transition from one to zero at the {\em critical} density, defined as the ratio of number of clauses to number of variables. For example, for random 3-SAT, the critical point was observed at density 4.26. Theoretical investigations into the location of random k-SAT have led to the precise identification of density for $k=2$ and existence of a sharp transition for large $k$~\cite{DSS15}. 

Furthermore, a phase-transition phenomenon has also been identified in
random $l$-XOR formulas (conjunctions of XOR constraints of length $l$), for $l \geq 1$ , without specifying an exact location for the phase-transition \cite{CDH03}. Pittel and Sorkin~\shortcite{PS16} identified the location of the phase transition for $l$-XOR formulas for $l > 3$. Dudek \textit{et al.}~\shortcite{DMV16} first studied the satisfiability threshold for the conjunction of random $k$-CNF and random variable-width XOR formulas. As is the case with random $k$-CNFs, experiments confirm that the hardest instances are at the critical threshold for $k$-CNF-XOR formulas~\cite{DMV17}. To the best of our knowledge, no prior work exists regarding the study of phase transition for a formula with one cardinality constraint in conjunction with a set of random variable-width XOR constraints (1-CARD-XOR formulas). 

\subsubsection{Cardinality Encodings}
Cardinality (CARD) constraints  naturally arise in many different contexts, such as computer tomography~\cite{GGP99}, MaxSAT algorithms~\cite{FM06}, radio frequency assignment~\cite{YD13}, product configuration~\cite{YD13}, program repair~\cite{joshikroening-fm15} and weighted counting problems~\cite{Par18}. Due to their ubiquity in several application domains, several encodings have been developed which translate them into the Boolean CNF form such as the Totalizer encoding~\cite{totalizer}, the Sequential counter~\cite{Sinz05}, Adder~\cite{ES06}, BDD based encoding~\cite{bdd},  Cardinality Networks~\cite{CardinalityNA09}, and the like. These encodings exhibit different characteristics in terms of their size (e.g., number of clauses and number of variables) and whether they preserve arc-consistency, i.e., the solver is able to detect inconsistencies by unit propagation alone~\cite{ZY00}. Therefore, we focus on observing the repeatability of behavior across different encodings before drawing a conclusion in our study. 
