

\section{Establishing a Phase-Transition}
\label{sec:analysis}
\renewcommand{\theenumi}{(\alph{enumi})}
In this section we will first define and show the existence of a phase transition phenomenon in 1-CARD-XOR formulas. A phase transition boundary is defined by a function $\phi$ such that, w.h.p., a random formula with $s<\phi$
is satisfiable, and becomes unsatisfiable as soon as $s>\phi$.
Note that the random variable denoting a 1-CARD-XOR formula is $\FQ = \F \wedge \Q$.

\begin{theorem}
	\label{thm:existence}
	 (\cite{DMV16}, Theorem 1)

	If $\phi{(k/n)} = \frac{1}{n} \log_2{(\#\F)}$, then for all $k\geq1$ and $s\geq0$:
	\begin{enumerate}
	 \item If $s < \phi{(k/n)} $, then w.h.p. $\FQ$ is satisfiable.
	 \item If $s > \phi{(k/n)} $, then w.h.p. $\FQ$ is unsatisfiable.
	\end{enumerate}
\end{theorem}
	
This is a special case of Theorem 1 given in \cite{DMV16}, which establishes the existence of a phase transition for random CNF-XOR formulas.
 %Part(a) is proven in Lemma \ref{lemma:lowerbound} and part(b) in Lemma \ref{lemma:upperbound}.
 The region of satisfiability is sharply separated from the region of unsatisfiability by the function $\phi{(k/n)}$.  
Since we give an explicit function $\phi$ in Lemma \ref{lemma:lowerbound} and \ref{lemma:upperbound}, we are able to show sharp numerical bounds on the phase transition boundary. We plot the transition function, $\phi{(k/n)}$, with a red line in all figures in this paper.

Next, we use a result from \cite{DMV16}, which gives us the probability of a CNF being satisfiable when conjuncted with some number of XOR constraints, in terms of the count of solutions of that CNF. Using Theorem \ref{thm:satisfiability:} we relate the satisfiability threshold with the model count of any formula when conjuncted with \textit{random} XORs.           
 

\begin{theorem}
	\label{thm:satisfiability:}
     (\cite{DMV16} , Lemma 7 and 12)
     
      Let $\alpha\geq 1$, $s \geq 0$, $n \geq 0$, and let F be a formula defined over $\{ X_1,\ldots,X_n \}$. Then 
	\begin{enumerate}
		
	 \item $\P{ F \wedge \Q  \text{ is satisfiable}   \bigm\vert \#F \geq 2^{\ceil{sn} +\alpha} }$ $ \geq 1- 2^{-\alpha}$.
	 \item $\P{F \wedge \Q \text{ is unsatisfiable} \bigm\vert \#F \leq 2^{\ceil{sn} -\alpha} } \geq 1- 2^{-\alpha}$.
	\end{enumerate}
\end{theorem}

Since the bounds given in Theorem \ref{thm:existence} are not in closed form, we provide analytic bounds which are weaker. The separation in the lower and upper bound is exactly 1 for $ k> \floor{n/2}$ while for  $k \leq \floor{n/2}$ the separation is $\bigO(\log(n)/n)$


\begin{theorem}
	\label{thm:bounds}$\FQ$ is satisfiable w.h.p. if:
	\begin{enumerate}
		\item $s <  H(k/n) - \log(8k(1-k/n))/n$, and $0<k<n/2$
		\item $s < 1 - 1/n $, and $n/2\leq k\leq n$  
	\end{enumerate}
	$\FQ$ is unsatisfiable  w.h.p. if:
	\begin{enumerate}
		\setcounter{enumi}{2}
		\item If $s >  H(k/n) $, and $0<k<n/2$
		\item If $s > 1  $,  and $n/2\leq k\leq n$ 
	\end{enumerate}
\end{theorem}
\begin{proof}
Part(a) and (b) follow from Lemma \ref{lemma:lowerbound:approx} and Part(c) and (d) follow from Lemma \ref{lemma:upperbound:approx} presented in Sections \ref{sec:analysis}.1 and \ref{sec:analysis}.2 respectively.  
\newline
\end{proof}
We will use the facts that $\#\F = \sum_{w=0}^{k} {{n} \choose {w}}$ and $\FQ = \F \wedge \Q$.

%this bound is tighter 
%\begin{lemma}
%	\label{lemma:sum}
%	$\#\F < \delta \cdot 2^{nH(k)}$ where $\delta = 0.98$
%\end{lemma}
%\begin{proof}
%	The number of solutions for the constraint $\F$ is the number of vectors of length $n$ and Hamming weight no more than $k$, $\#\F = \sum_{w=1}^{k} {{n} \choose {w}}$. 
	
%	Using the bound proven as Lemma 5 in \cite{GKM12}, we get that		
%	$ \sum_{w=1}^{k} {{n} \choose {w}} < \delta \cdot 2^{nH(k)}$ where $\delta = 0.98$, and $H(k) = -k\log_{2}{k}-(1-k)\log_{2}{(1-k)}$. 
%\end{proof}
We use a commonly known bound for summation of binomial coefficients, 
\begin{lemma}(\cite{MS78}, Lemma 10.8). \newline
$  2^{nH(k/n)}/ (8k(1-k/n)) \leq  \sum_{w=0}^{k} {{n} \choose {w}} \leq 2^{nH(k/n)}$, \newline for $0< k\leq n/2$ and for all $n \geq 1$.
\end{lemma}	


\subsection{Lower bound}
\begin{lemma}
	\label{lemma:lowerbound}	
	Let $k \geq 2$ and $ s\geq 0.$ If $s < \frac{1}{n} {log_{2} \#\F }$ as $\lim\limits{n \to \infty}$, then w.h.p. $ \FQ$ is satisfiable.
\end{lemma}
\begin{proof}	Since $2^{s}  < \#\F^{1/n}$ we can choose $\delta > 0$ and $ N > 0$ such that $ 2^{s + \delta + 1/N} < \#\F^{1/n}$. We can always find a small enough $\delta$ and a sufficiently large $N$ such that this is true. Since we are concerned with behavior asymptotic in $n$, we consider only $n>2N$. Then we have $2^{sn + \delta n + 2} < \#\F$ and so $2^{\ceil{sn} + \delta n + 1} < \#\F$. Let $\alpha = \delta n + 1$, so we get $2^{\ceil{sn}+\alpha} \leq  \#\F.$  Using Theorem \ref{thm:satisfiability:}a we see that $\P{\FQ\text{ is SAT} \bigm\vert \#\F\geq2^{\ceil{sn} + \alpha} } \geq 1- 2^{-\alpha}$. Since $\lim \limits_{n \to \infty} 1-2^{-\delta n - 1}$ converges to 1, $\FQ$ is satisfiable w.h.p.
\end{proof}


\begin{lemma}
	\label{lemma:lowerbound:approx}
	For $s \geq 0 $ and $ 0 < k < n$ and $\lim\limits{n \to \infty}$, $\FQ$ is satisfiable w.h.p. if:
	\begin{enumerate}
	\item $k\leq n/2 $ and $ s <  H(k/n) - \log(8k(1-k/n))/n $
	\item $k> n/2$ and $ s<1-1/n $
	\end{enumerate}
\end{lemma}
\begin{proof}
  For $n/2 < k \leq n$ , $\#\F > 2^{n-1}$. Observing that $s<1-1/n<\frac{1}{n} \log_{2}\#\F$ we see that Part(b) is an immediate consequence of Lemma \ref{lemma:lowerbound}.	
 
	Using the bound shown in Lemma , for $0< k<n/2$ we get that		
. Since $2^{s}  < 2^{H(k/n) - \log(8k(1-k/n))/n}$ we can choose $\delta > 0$ 	and $ N > 0$ such that $ 2^{s + \delta + 1/N} < 2^{H(k/n) - \log(8k(1-k/n))/n}$. We can always find a small enough $\delta$ and a sufficiently large $N$ such that this is true. Since we are concerned with behavior asymptotic in $n$, we consider only $n>2N$. Then we have $2^{sn + \delta n + 1} < 2^{nH(k/n) - \log(8k(1-k/n))}$ and so $2^{\ceil{sn} + \delta n + 1} < 2^{nH(k/n) - \log(8k(1-k/n))}$. Let $\alpha = \delta n + 1$, so we get $2^{\ceil{sn}+\alpha} < 2^{nH(k/n) - \log(8k(1-k/n))}.$  Using Theorem \ref{thm:satisfiability:}a we see that $\P{\FQ\text{ is SAT} \bigm\vert 2^{nH(k/n) - \log(8k(1-k/n))}\geq \right.$ $\left.2^{\ceil{sn} + \alpha} } \geq 1- 2^{-\alpha}$. Since $\lim \limits_{n \to \infty} 1-2^{-\delta n - 1}$ converges to 1, $\FQ$ is satisfiable w.h.p.
\end{proof}



\subsection{Upper bound}
\begin{lemma}
	\label{lemma:upperbound}	
	Let $k \geq 2, s\geq 0,$ and $r\geq 0$. If $s > \lim\limits_{n \to \infty} \frac{1}{n}{log_{2}\#\F}$, then w.h.p. $ \FQ$ is unsatisfiable.
\end{lemma}
\begin{proof}	Since $2^{s}  > \#\F^{1/n}$ we can choose $\delta > 0$ 	and $ N > 0$ such that $ 2^{s - \delta - 1/N} > \#\F^{1/n}$. We can always find a small enough $\delta$ and a sufficiently large $N$ such that this is true. Since we are concerned with behavior asymptotic in $n$, we consider only $n>N$. Then we have $2^{sn - \delta n -1} > \#\F$ and so $2^{\ceil{sn} - \delta n - 1} > \#\F$. Let $\alpha = \delta n + 1$, so we get $2^{\ceil{sn}-\alpha} > \#\F.$  Using Theorem \ref{thm:satisfiability:}b we see $\P{\FQ\text{ is UNSAT} \bigm\vert \#\F\geq2^{\ceil{sn} - \alpha} } \geq 1- 2^{-\alpha}$. Since $\lim \limits_{n \to \infty} 1-2^{-\delta n - 1}$ converges to 1, $\FQ$ is satisfiable w.h.p.
\end{proof}
\begin{lemma}
	\label{lemma:upperbound:approx}
	For $s \geq 0 $ , $ 0 \leq k\leq n$ and $\lim\limits{n \to \infty}$, $\FQ$ is unsatisfiable w.h.p. if:
	\begin{enumerate}
	\item $k<n/2 $ and $ s \geq  H(k/n) $
	\item $k\geq n/2$ and $ s > 1 $
	\end{enumerate}
\end{lemma}
\begin{proof}
 For $n/2 \leq k < n$ observing that $s>1>\frac{1}{n} \log_{2}\#\F$ we see that Part(b) is an immediate consequence of Lemma \ref{lemma:upperbound}.

	 Since $2^{s}  > 2^{H(k/n)}$ we can choose $\delta > 0$ 	and $ N > 0$ such that $ 2^{s - \delta - 1/N} > 2^{H(k/n)}$. We can always find a small enough $\delta$ and a sufficiently large $N$ such that this is true. Since we are concerned with behavior asymptotic in $n$, we consider only $n>N$. Then we have $2^{sn - \delta n -1} > 2^{nH(k/n)}$ and so $2^{\ceil{sn} - \delta n - 1} > 2^{nH(k/n)}$. Let $\alpha = \delta n + 1$, so we get $2^{\ceil{sn}-\alpha} > 2^{nH(k/n)}.$  Using Theorem \ref{thm:satisfiability:}b we see that $\P{\FQ\text{ is UNSAT} \bigm\vert 2^{nH(k/n)}\geq2^{\ceil{sn} - \alpha} } \geq 1- 2^{-\alpha}$. Since $\lim \limits_{n \to \infty} 1-2^{-\delta n - 1}$ converges to 1, $\FQ$ is satisfiable w.h.p.
\end{proof}